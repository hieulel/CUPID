\documentclass{beamer}

\usetheme{Madrid}

\title[CUPID 101]
{CUPID's Output}
\subtitle{How to get started}
\author{Hieu Le}
\institute{UofA}

\usepackage{hyperref}

\AtBeginSection[]
{
  \begin{frame}
    \frametitle{Table of Contents}
    \tableofcontents[currentsection]
  \end{frame}
}

\begin{document}

\frame{\titlepage}



\section{Introduction}
\begin{frame}{Purpose}
	\begin{itemize}
	\item From the CUPID's sensitivity analysis, I guess you are familiarise how CUPID's input influence the the output
	\item Here, I am presenting you two methods making the output becomes prettier
	\end{itemize}
\end{frame}

\begin{frame}{Two methods}
	It comprises of two methods:
		\begin{enumerate}
			\item \texttt{cup2rdb}
			\item \texttt{cuprd3}
		\end{enumerate}
	\texttt{cup2rdb} will be relatively easier than \texttt{cuprd3}
\end{frame}

\begin{frame}{Before you get started\ldots}
	Before you even start this process, you must have these tools on your computer.
	\begin{enumerate}
		\item Terminal (MacOS or Linux) or Command Line, known as CMD (WindowOS)
		\item \texttt{gfortran} (I installed this on your computer previously)
		\item CUPID folder made by Hieu Le
	\end{enumerate}
\end{frame}

\begin{frame}{How to check - Terminal or CMD}
	\begin{enumerate}
		\item Search for "Terminal" or "CMD" and open it
		\item Type \texttt{gfortran --version} to show the version
		\footnote{On WindowOS, type "\texttt{bash}" to open UNIX environment before implementing this step. Meanwhile, on MacOS and Linux, you don't need to do this.}
	\end{enumerate}
	Now thing get a bit complicated, I will demonstrate this on the video\ldots
\end{frame}

\section{cup2rdb}
	\begin{frame}{\texttt{cup2rdb}}
		The first type of compiler I want to mention is \texttt{cup2rdb}. 
		
		The command line construct like this
		
		\texttt{
		./cup2rdb cupfile.rd outputfile.org | somecondition \ldots
		}
		
		The output can be further strip down and seen in \url{https://soils.wisc.edu/facstaff/wayne/cupid/cup2rdb.html}
	\end{frame}
	
	\begin{frame}{Trimmed output}
			\begin{table}[h]
			\begin{tabular}{lllllll}
DOY    & Time     & Layer    & Depth  & Soil water potential & Water content \\
\hline
222    & 26     & 1      & 2.00   & 17.60          & 43.19         \\
222    & 26     & 2      & 1.60   & 17.96          & 43.62         \\
222    & 26     & 3      & 1.30   & 18.33          & 44.04         \\
222    & 26     & 4      & 1.00   & 18.85          & 44.61         \\
222    & 26     & 5      & 0.70   & 19.72          & 45.53         \\
\vdots    & \vdots & \vdots & \vdots & \vdots & \vdots         
		\end{tabular}
		\caption{Strip output sample}
		\end{table}
	\end{frame}
	
\section{cuprd3}
	\begin{frame}{\texttt{cuprd3}}
	The second method requires a bit more involvement of command line, meaning you have to work with Terminal and CMD
	
	This has been well documented in CUPID's documentation \texttt{cuprd3.docx}
	
	\end{frame}

\end{document}